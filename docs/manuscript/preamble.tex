% Any extra LaTeX you need in the preamble

In species that provide parental care, successful rearing of offspring involves a shift from aggressive and sexual behaviors to more caring and nurturing ones.  We hypothesize that these shifts in behavior are related to changes in gene expression that are controlled by internal mechanisms rather than by the external cues.  We found that gene expression in the hypothalamus is fairly insensitive to the natural transition between egg and nestling stages; however, in the pituitary and gonads, gene expression showed much more plasticity over the course of reproduction. We show evidence that natural transitions in reproductive and parental behavior by mediated internal clock controlling gene expression in the pitary and gonads but that the hypothalamus is very sensitive to external cues that signal a distruption in the natural parental care cycle. This research provides a deeper understanding of the interplay of genes, hormones, and parental care and has important implications for understanding molecular processes that underlie behavior transitions. 